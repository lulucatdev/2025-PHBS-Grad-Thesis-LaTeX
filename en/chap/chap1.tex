\chapter{Introduction}

\section{Research Background}

In modern organizational management practices, the phenomenon of employee \enquote{shirking} (also referred to as \enquote{loafing on the job} or \enquote{work avoidance}) is increasingly attracting attention. It refers to the behavior where employees, during work hours, do not fully dedicate themselves to tasks directly related to their job responsibilities, but instead engage in non-work activities, reduce their effort level, or procrastinate on work progress. This phenomenon is not merely a matter of individual laziness but is rooted in the complex organizational environment and employment relationship, particularly where {information asymmetry} commonly exists between the employer (principal) and the employee (agent) \citep{akerlof1970market, spence1973job}. Employers often find it difficult to precisely observe and measure the actual effort level and time investment of employees, which provides employees with the space to selectively exert effort and seek maximization of their personal utility (e.g., pursuing leisure, handling private affairs), thereby engaging in some degree of \enquote{shirking} behavior \citep{alchian1972production}.

Although the phenomenon of \enquote{shirking} is widespread, its underlying economic principles, particularly how employees make \enquote{optimal} shirking decisions in an environment of asymmetric information, and how these decisions interact with corporate incentive mechanisms and monitoring strategies, have not yet been fully and systematically explored. Existing literature in management and organizational behavior often analyzes it from psychological, cultural, or managerial technique perspectives \citep[e.g.,][]{ashforth1990social, robbins2016organizational}. While economics literature contains extensive research on principal-agent models and incentive theory \citep{holmstrom1979moral, grossman1983analysis}, refining these theories to analyze the prevalent and variably-intense \enquote{shirking} behavior found in everyday work scenarios, explicitly exploring its \enquote{optimality} issue, and determining the employer's optimal response strategy, still leaves significant room for research. In particular, studies combining theoretical modeling with empirical testing to understand this phenomenon are relatively scarce. Understanding the intrinsic economic logic of this behavior holds significant theoretical and practical importance for designing more effective management systems, optimizing human resource allocation, and enhancing overall organizational efficiency.

This study attempts to approach the issue from the perspectives of {information economics} and {game theory}, viewing \enquote{shirking} as a rational (or boundedly rational) decision-making process by employees facing incomplete information and specific incentive constraints. We aim to construct a theoretical framework to analyze how employees weigh the costs of exerting effort, the risk of being detected and punished, and the utility derived from \enquote{shirking}, thereby choosing an effort level that is optimal for themselves. Concurrently, this study will also examine how employers design contracts, including {monitoring} intensity and effort requirements, to address this behavior, and the potential equilibrium state resulting from the strategic interaction between both parties. Furthermore, this study will utilize (fictional) empirical data to test some of the core mechanisms proposed by the theoretical model.

\section{Research Questions}

Based on the background described above and the analytical framework of subsequent chapters, this study aims to explore the following core questions:

\begin{enumerate}
    \item Under conditions of asymmetric information where the employer cannot fully observe the employee's effort level, how is the employee's {optimal shirking} strategy (manifested as the choice of effort level in this simplified model) determined? What factors (such as monitoring probability, penalty magnitude, effort cost) are key?
    \item How does the employer design the optimal {monitoring} and {incentive} contract (manifested as the minimum effort requirement $e_{min}$, monitoring probability $p$, and fixed wage $w$ in this simplified model) to address potential employee {shirking} behavior and maximize profits? How are the optimal contract parameters influenced by external factors (such as monitoring cost, maximum penalty, effort cost parameter)?
    \item (Based on fictional empirical analysis) In a realistic (simulated) environment, what is the relationship between the firm's monitoring intensity, the incentive strength of performance pay, the characteristics of the work task itself (e.g., complexity), and the employee's {shirking} behavior? Do the theoretical predictions find support in the (fictional) data?
\end{enumerate}

\section{Research Contributions}

This study is expected to make contributions in the following aspects:

First, at the {theoretical level}, by applying classic {principal-agent theory} and {game theory} models to analyze \enquote{shirking}, a specific and common phenomenon in the modern workplace, this study deepens the understanding of employee opportunistic behavior under asymmetric information. By explicitly exploring the formation mechanism of the \enquote{optimal shirking} strategy (a binary choice between meeting the minimum standard or zero effort in this model) and its response to monitoring and punishment, it provides analytical perspectives for labor economics and organizational economics.

Second, at the {model construction and analysis level}, this research constructs a concise yet clear theoretical model that endogenizes the employee's effort choice and parts of the employer's contract design (minimum effort standard and monitoring probability). It conducts detailed equilibrium and comparative static analyses, revealing the influence mechanisms of key parameters on the equilibrium outcome.

Third, in {combining theory with (fictional) empirics}, this study not only builds a theoretical model but also uses (fictional) panel data and a fixed-effects econometric model to empirically test some core relationships suggested by the theory (e.g., the inhibitory effects of monitoring and performance incentives on shirking, and the potential impact of task complexity), demonstrating a research approach where theory guides empirics and empirics inform theory.

Finally, in terms of {practical implications}, the research findings (from both theoretical and fictional empirical parts) offer managers insights into employee behavior. Understanding the economic roots and influencing factors of \enquote{shirking} helps managers design more effective monitoring strategies and incentive mechanisms (e.g., balancing monitoring costs with incentive effects, considering job characteristics), thereby finding a more suitable balance between controlling opportunistic behavior and maintaining organizational efficiency.

\section{Structure of the Thesis}

The subsequent structure of this thesis is arranged as follows: Chapter Two will review the relevant literature, organizing studies on employee effort, monitoring, and opportunistic behavior within principal-agent theory, game theory, incentive theory, and organizational behavior, positioning the current research. Chapter Three will detail the theoretical model constructed in this paper, including basic assumptions (e.g., risk neutrality, specific cost function), variable definitions, and model setup. Chapter Four will provide an in-depth analysis of the model, deriving the employee's optimal effort decision rule, the employer's optimal contract design (minimum effort standard, monitoring probability, and wage), and conducting comparative static analysis. Chapter Five will present the empirical study based on (fictional) China Enterprise Employee Survey data, including model specification, variable descriptions, baseline regression results, and robustness discussions, testing some predictions of the theoretical model. Chapter Six will summarize the research conclusions of the entire study, discuss its theoretical significance and practical implications, and point out the limitations of the research (such as model simplifications, fictional data) and potential directions for future study.