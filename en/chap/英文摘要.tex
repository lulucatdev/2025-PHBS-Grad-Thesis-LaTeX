\begin{eabstract}
	\addcontentsline{toc}{chapter}{ABSTRACT}

	In the modern workplace, the phenomenon of \enquote{shirking} (also known as \enquote{moyü} or work avoidance), defined as employees engaging in non-work-related activities or reducing work effort during work hours, is increasingly common. This behavior not only affects organizational efficiency but also reflects profound economic issues within the employment relationship. This paper explores the economic principles behind {shirking} from the perspective of {information economics}, based on {principal-agent theory} and {game theory}. It investigates how employees choose optimal {shirking} strategies under conditions of {information asymmetry}, and how employers design optimal {incentive} and {monitoring} mechanisms in response. The research aims to reveal the underlying economic logic of this behavior and provide theoretical insights and (simulated) empirical evidence for understanding and managing modern labor relations.

	This study first constructs a concise theoretical model. In this model, a risk-neutral employer cannot directly observe the effort level of a risk-neutral employee but can incentivize the employee by setting a minimum effort standard, paying a fixed wage, and implementing a probabilistic {monitoring} and punishment mechanism. The model derivation shows that the employee's optimal decision exhibits a threshold effect: the employee chooses to meet the minimum standard only if the expected penalty (monitoring probability multiplied by penalty magnitude) exceeds the cost of achieving that standard; otherwise, the employee chooses zero effort (maximizing shirking). Furthermore, the model analyzes how the employer, balancing output gains, wage payments, and monitoring costs, selects the optimal minimum effort requirement and monitoring probability.

	To test the implications of the theoretical model, this paper conducts an empirical analysis using (fictional) panel data from the \enquote{China Enterprise Employee Survey (CEES)} and employing a fixed-effects econometric model. The results indicate that increased enterprise {monitoring} intensity and a higher share of {performance pay} are significantly negatively correlated with the employee shirking index, confirming the effectiveness of {monitoring} and {incentives} in curbing opportunistic behavior. Conversely, higher task complexity is positively associated with the shirking index, possibly reflecting the difficulty of effectively monitoring complex tasks.

	Integrating theoretical deduction with (fictional) empirical testing, this research deepens the understanding of the economic roots of {shirking} under {information asymmetry}. The findings offer implications for managers in designing more effective {incentive} mechanisms (e.g., balancing performance pay and monitoring investments) and management strategies (e.g., adopting differentiated approaches for tasks of varying complexity). This paper emphasizes that understanding the economics of {shirking} does not endorse the behavior but aims to promote more harmonious and efficient labor relations through scientific analysis.
\end{eabstract}

% vim:ts=4:sw=4
