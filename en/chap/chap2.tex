\chapter{Literature Review}

This chapter aims to systematically review the theoretical and empirical research related to employee \enquote{shirking} behavior, laying the groundwork for subsequent model construction and analysis. We will primarily focus on reviewing literature concerning principal-agent theory, incentive and contract theory, the application of game theory in employment relationships, and relevant studies in organizational behavior regarding employee motivation and opportunism.

\section{Principal-Agent Theory and Information Asymmetry}

One of the core features of modern corporate organizations is the separation of ownership and control, which forms the basis of the principal-agent relationship \citep{jensen1976theory}. In this relationship, the principal (e.g., employer, shareholder) delegates decision-making authority or task execution power to the agent (e.g., employee, manager), expecting the agent to act in the best interest of the principal. However, due to information asymmetry, particularly the difficulty for the principal to fully observe and verify the agent's actions (such as effort level), the agent may leverage this informational advantage to pursue their own interests, leading to the problem of Moral Hazard \citep{holmstrom1979moral}.

Employee \enquote{shirking} behavior can be considered a typical manifestation of moral hazard. Because employers cannot precisely monitor every activity and effort input of their employees, employees may choose to exert less effort than agreed upon or expected, use work time for non-work matters, or complete tasks with lower efficiency, i.e., \enquote{Shirking} on the job \citep{alchian1972production}. Akerlof's \enquote{market for lemons} theory \citep{akerlof1970market} and Spence's signaling theory \citep{spence1973job} also reveal how information asymmetry affects market efficiency and individual behavior, principles that equally apply within the labor market. To mitigate problems arising from information asymmetry, principals need to design mechanisms to monitor agent behavior or incentivize effort exertion.

Early principal-agent models typically assumed that the agent's effort is unobservable, and the principal can only observe an output signal related to effort, which is also subject to random disturbances \citep{holmstrom1979moral, grossman1983analysis}. This made incentive contracts based on output the focus of research. Studies show that optimal contract design requires a trade-off between risk sharing and providing incentives. When agents are risk-averse, transferring excessive risk to them reduces their utility and requires a higher risk premium; insufficient incentives, however, lead agents to choose suboptimal effort levels.

\section{Incentive Theory and Contract Design}

To address the agency problem, economists have developed a rich body of incentive theory. The core idea is to align the agent's interests with the principal's objectives through the design of effective contracts.

Performance Pay is one of the most widely applied incentive tools. Theoretical research and empirical evidence suggest that linking compensation to measurable performance indicators can significantly increase employee effort and productivity \citep{lazear1999performance}. However, designing performance pay faces numerous challenges, such as: the measurability issue of performance indicators (output of some tasks is difficult to quantify), the multi-task problem (employees may focus only on tasks with easily measured performance while neglecting other important responsibilities), and potential inducements for short-term behavior and excessive competition \citep{holmstrom1991multitask}.

Efficiency Wage Theory offers another perspective. It posits that employers paying wages above the market-clearing level can increase the opportunity cost of shirking for employees (i.e., the cost of losing a high-paying job), thereby motivating them to work harder and reducing the need for supervision \citep{shapiro1984equilibrium}. Efficiency wages can also attract higher-quality employees, reduce turnover rates, and enhance employee morale and sense of fairness \citep{akerlof1986efficiency}.

Furthermore, non-monetary incentives such as promotions, career development, and reputation mechanisms also play important roles in the employment relationship \citep{fama1980agency, gibbons1999careers}. Long-term employment relationships, internal labor markets, and corporate culture, by fostering trust and repeated interactions, can mitigate short-term opportunistic behavior to some extent.

The development of contract theory has also gradually moved from assumptions of perfect rationality and complete contracts towards acknowledging the reality of bounded rationality and contract incompleteness \citep{hart1995firms}. The concept of the Psychological Contract emphasizes the implicit expectations and mutual obligations in the employment relationship that are not explicitly written into formal contracts \citep{rousseau1995psychological}. When employees perceive that the organization has violated the psychological contract (e.g., broken promises, unfair treatment), their work motivation, loyalty, and effort levels may significantly decrease, making them more prone to negative behaviors such as \enquote{shirking}.

\section{Game Theory Perspective on the Employment Relationship}

The employment relationship can be viewed as an ongoing game between the employer and the employee. Both parties, operating under conditions of incomplete information, choose their optimal actions based on expectations of the other party's strategies.

Incorporating \enquote{shirking} behavior into a game-theoretic framework helps analyze the interaction of strategies and equilibrium outcomes. For instance, the employer's monitoring strategy (e.g., frequency, intensity) and the employee's shirking strategy (e.g., degree, method) can be treated as interdependent decision variables. Increased monitoring investment by the employer can raise the probability of detecting shirking, thus deterring such behavior, but monitoring itself is costly. The employee, in turn, must weigh the utility gained from shirking against the risk of detection and the cost of working diligently. Tirole's (1986) \nocite{tirole1986hierarchies} research on collusion within organizations also suggests that monitoring systems themselves may have loopholes, potentially allowing for some form of \enquote{collusion} between managers and employees that affects monitoring effectiveness.

Repeated game models are particularly suitable for analyzing long-term employment relationships. In repeated games, reputation mechanisms and retaliatory strategies (such as \enquote{trigger strategies}) can support cooperative equilibria, where employees choose not to shirk, and employers choose to trust or monitor less \citep{axelrod1984evolution}. However, maintaining cooperative equilibrium requires certain conditions, such as a sufficiently long game duration (or low probability of termination), sufficient patience from both parties (discount factor not too low), and adequate information transparency.

\section{Organizational Behavior and Psychological Perspectives}

Economic models typically assume individuals are rational and self-interested, seeking to maximize utility. In contrast, organizational behavior and psychology offer richer explanations for employee motivation and behavior.

Besides external incentives (like wages, bonuses), Intrinsic Motivation—the enjoyment, sense of achievement, autonomy, etc., derived from the work itself—is also a significant driver of employee effort \citep{deci1985intrinsic}. Over-reliance on external controls and monitoring can sometimes undermine employees' intrinsic motivation, leading to the so-called \enquote{crowding-out effect} of incentives \citep{frey1997not}.

Factors such as Burnout, Organizational Justice, Leadership Style, and Organizational Culture have also been shown to be closely related to employee work attitudes and behaviors (including work engagement, absenteeism, turnover intention, and counterproductive work behaviors like \enquote{shirking}) \citep{maslach2001job, colquitt2001organizational}. For example, when employees perceive distributive or procedural injustice, they might \enquote{correct} this perceived unfairness by reducing effort or increasing non-work activities. \enquote{Cyberloafing}, the use of company-provided internet access for non-work-related online activities during work hours, has emerged as a new research hotspot in modern workplaces \citep{lim2002it}.

\section{Literature Summary and Research Positioning}

In summary, the existing literature has explored the theoretical foundations and influencing factors related to employee \enquote{shirking} behavior from various disciplinary perspectives. Principal-agent theory reveals the roots of information asymmetry and moral hazard; incentive theory examines how to guide employee behavior through contract design; game theory analyzes the strategic interactions between employers and employees; and organizational behavior emphasizes the role of psychological factors and the organizational environment.

However, current research still exhibits some limitations:
1.  Most economic models tend to treat effort/shirking as a discrete choice (e.g., effort/no effort) or focus on specific types of shirking (e.g., reducing output quantity/quality), with fewer models formalizing the decision-making process where employees choose the *degree* of \enquote{shirking} continuously or multi-dimensionally.
2.  Research that endogenizes both the employee's individual \enquote{optimal shirking} decision and the employer's optimal incentive/monitoring strategy within the same theoretical framework, analyzing their interactive equilibrium, is relatively scarce.
3.  There is a lack of fine-grained theoretical characterization regarding how the \enquote{degree} of \enquote{shirking} is determined, i.e., how employees seek an optimal balance between pursuing personal leisure utility and maintaining job security/avoiding punishment.

Building upon previous research, this study attempts to focus on the issue of employee \enquote{optimal shirking} strategy choice in an environment of asymmetric information. We will construct a theoretical model that explicitly treats the degree of \enquote{shirking} as a continuous decision variable for the employee, analyzing how it is influenced by factors such as wage structure, monitoring probability, punishment severity, job characteristics, and individual preferences. Concurrently, the model will also incorporate the employer's incentive and monitoring strategies as endogenous variables to examine the equilibrium outcome under the strategic interaction of both parties. The aim is to provide a more refined economic analysis framework for understanding the ubiquitous phenomenon of \enquote{shirking} in the modern workplace and to offer theoretical insights for designing more effective management strategies.