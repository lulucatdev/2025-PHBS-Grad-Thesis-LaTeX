\chapter{Conclusion and Outlook}
\label{chap:conclusion}

This study, taking the prevalent phenomenon of employee \enquote{shirking} (work avoidance) in modern workplaces as its starting point, has attempted to construct theoretical models and combine them with (fictional) empirical analysis from the perspectives of information economics and game theory. The goal was to explore the formation mechanism and influencing factors of employees' \enquote{optimal shirking} strategies under asymmetric information, as well as the optimal response strategies of employers. Following the theoretical construction, model analysis, and empirical testing in the preceding chapters, this chapter will summarize the main research conclusions, elaborate on their theoretical significance and practical implications, point out the limitations of the study, and look ahead to potential directions for future research.

\section{Summary of Research Findings}

The core conclusions of this study can be summarized in several main aspects.
First, at the theoretical level, the research reveals the nature of \enquote{shirking} as a rational choice under asymmetric information. The constructed principal-agent model shows that when employers cannot fully observe employee effort, employees tend to choose an effort level that maximizes their own utility, balancing wage benefits, effort costs, and the leisure utility or value of engaging in private affairs derived from \enquote{shirking}. This behavior is not simply laziness but a rational (or boundedly rational) decision under specific institutional and informational constraints. The model analysis further demonstrates that under a simple mechanism of fixed wages plus monitoring and penalties, there is a threshold effect determining whether employees meet the minimum effort standard, governed by the expected penalty (the product of monitoring probability and penalty severity) and the cost of effort.

Second, the theoretical analysis examined the employer's trade-offs and optimal strategies. Facing potential employee \enquote{shirking}, a risk-neutral employer aims to maximize profit by designing contracts (including wage structure, monitoring intensity, effort standards, etc.). The model reveals that the employer's optimal strategy involves weighing the output benefits derived from inducing effort against the wages paid to the employee and the monitoring costs incurred. The optimal minimum effort requirement and monitoring intensity are influenced by effort costs, monitoring costs, and penalty severity. For instance, higher monitoring or effort costs prompt employers to lower effort requirements, whereas greater penalty power inclines them to set higher effort targets.

Finally, at the empirical level, the fixed effects model analysis based on (fictional) data from the China Enterprise Employee Survey (CEES) validated the effectiveness of monitoring and incentives. The results show a significant negative correlation between the monitoring intensity implemented by firms and the employees' \enquote{shirking} index, confirming the deterrent effect of monitoring and indicating that strengthening supervision is an effective means to curb work avoidance behavior. Simultaneously, the share of performance pay is also significantly negatively correlated with the \enquote{shirking} index, suggesting that linking compensation more closely to performance effectively motivates employees to reduce non-work activities. Interestingly, task complexity (and autonomy) shows a significant positive correlation with the \enquote{shirking} index. This might imply that while complex and autonomous work can stimulate intrinsic motivation, its inherent difficulty in being monitored perhaps creates more opportunities for \enquote{shirking}, with the negative effect outweighing the positive motivational effect. These findings provide (simulated) empirical evidence for the theoretical model and reveal the differential impacts of various management tools on \enquote{shirking} behavior.

\section{Theoretical Significance}

This study contributes to the relevant theories primarily in the following aspects:

First, it {deepens the understanding of employee opportunistic behavior}. By incorporating the pervasive phenomenon of \enquote{shirking}—often viewed as irrational or purely a moral issue—into the analytical framework of information economics and rational choice, this study reveals its underlying economic logic as an outcome of individuals maximizing utility under asymmetric information and specific contractual constraints. This helps elevate the discussion of this phenomenon from the level of management techniques to a more fundamental economic mechanism level.

Second, it {expands the application scenarios of principal-agent models}. This research applies classic principal-agent theory to analyze the variably-intense \enquote{shirking} behavior in everyday work settings. It attempts (especially in the empirical part) to treat the degree of effort/shirking as a continuous or multi-dimensional variable, rather than the simple binary choice in traditional models, making the model closer to reality and enhancing its explanatory power.

Third, it {integrates multidisciplinary perspectives}. Although the core framework is economic, the study, particularly when discussing variables (like task complexity) and results, also draws upon insights from organizational behavior regarding intrinsic motivation and job design. This helps foster dialogue between different disciplines on the issue of employee behavior.

\section{Practical Implications}

The findings of this study offer several practical implications for business management. First, managers should acknowledge the economic roots of \enquote{shirking} and adopt a systematic approach to address it. Recognizing that employee \enquote{shirking} is related not only to individual attitudes but also to the firm's information environment, incentive mechanisms, and monitoring systems, relying solely on ideological education or severe punishment may have limited effect. Addressing the problem requires starting from institutional design. Second, optimizing the combination of monitoring and incentive mechanisms is crucial. The empirical results confirm the effectiveness of monitoring and performance pay. Firms should reasonably design monitoring strategies (e.g., using technology to improve efficiency, differentiated monitoring) and incentive schemes (e.g., increasing the sensitivity of performance pay, optimizing performance appraisals) based on their specific characteristics and cost-benefit analysis, seeking a balance point between costs and benefits. Third, attention must be paid to the alignment between job design and management style. The empirical finding on task complexity reminds managers that traditional tight monitoring might be unsuitable, or even encourage hidden \enquote{shirking}, for complex, creative jobs with high autonomy. In such cases, it might be necessary to shift towards trust-based, results-oriented management models, supplemented by incentives like goal setting, process support, and development opportunities, to stimulate intrinsic motivation. Finally, considering comprehensive governance is necessary. Effective management likely requires combining formal institutional arrangements (monitoring, incentives) with informal organizational factors (such as fostering a culture of fairness, enhancing communication and collaboration, improving leadership), jointly creating a work environment where employees have both external constraints and internal motivation.

\section{Research Limitations}

Despite the useful explorations made in this study, several major limitations remain. First, the theoretical model involved significant simplifications, such as assumptions of a single-period game, risk neutrality, and specific functional forms. This may limit the model's ability to fully capture complex realities, for instance, by not considering employee risk aversion, multi-task environments, or reputation and learning effects in long-term relationships. Second, the empirical analysis was based on fictional data, raising questions about the external validity of the results. Furthermore, in real research, accurately measuring \enquote{shirking} behavior itself is highly challenging: surveys are susceptible to subjective bias, while objective data might raise privacy concerns or fail to capture all forms of \enquote{shirking}. Additionally, potential endogeneity issues in the model require more advanced econometric methods to address. Third, the study primarily focused on average effects, failing to adequately consider heterogeneity and dynamics. Differences across various types of employees or types of \enquote{shirking} behavior were insufficiently explored, and the static model could not capture the dynamic evolution of behaviors and strategies over time. Fourth, the research predominantly viewed \enquote{shirking} as a negative phenomenon, neglecting its potential positive aspects, such as moderate relaxation helping to alleviate stress, restore energy, or even spark creativity.

\section{Directions for Future Research}

Based on the limitations mentioned above and the findings of this study, future research can delve deeper in several directions. First, richer theoretical models can be constructed, incorporating factors like employee risk aversion, multi-task agency, collusion and monitoring in team production, and long-term employment relationships within a dynamic game framework, to bring the models closer to real-world complexities. Second, the empirical research foundation should be strengthened by utilizing real, multi-source data, employing more reliable measurement methods, and applying more rigorous causal inference techniques (such as randomized controlled trials, regression discontinuity designs, improved difference-in-differences methods) to test theoretical predictions and identify key influencing factors. Third, deeper exploration of the heterogeneity of \enquote{shirking} behavior is needed, distinguishing between different types and motives of \enquote{shirking} and their varying impacts on performance, and examining how individual, job, and organizational characteristics moderate \enquote{shirking} behavior and its consequences. Fourth, attention should be paid to the impact of emerging technologies and work models. Research should investigate how digital transformation, AI monitoring, remote work, and the gig economy are changing the forms, motives, and management responses related to \enquote{shirking}. Fifth, cross-cultural comparative studies could be conducted to explore how social norms, values, and legal systems in different cultural contexts affect the acceptance of \enquote{shirking} behavior and corporate management practices. In conclusion, employee \enquote{shirking} is a complex and important management phenomenon. This study's preliminary exploration from an economic perspective hopes to lay a foundation for subsequent, more in-depth and comprehensive research, and to provide valuable insights for managers tackling this challenge.