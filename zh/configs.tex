% 需替换 版权声明.PDF 为门户下载的PDF
% 然后修改 configs.tex 中的信息
% 修改完成后,把本页信息复制粘贴到 1_en/configs.tex 和 2_zh/configs.tex 中

\newcommand{\zhtitle}{\zhquote{摸鱼}的经济学原理——信息不对称下的最优偷懒策略}
\newcommand{\entitle}{The Economics of \enquote{Idleness} - Optimal Laziness under Asymmetric Information}
\newcommand{\zhauthor}{李嘉图}
\newcommand{\enauthor}{Li Jiatu}
\newcommand{\thestudentid}{2201212000}
\newcommand{\zhmajor}{金融硕士} % 专业:金融硕士/西方经济学/企业管理/新闻与传播硕士
\newcommand{\enmajor}{Master of Finance} % 专业的英文名:Master of Finance/Master of Economics/Master of Management/Master of Journalism and Communication
\newcommand{\majordirection}{金融管理} % 研究方向:金融管理/数量金融/金融科技/金融投资/财经传媒(其他专业空着)
\newcommand{\titlelines}{3} % 你的论文题目需要几行
\newcommand{\theyear}{2025} % 论文年份
\newcommand{\themonth}{5} % 论文月份,具体时间以教务为准,初稿3月,送审4月,答辩5月,最终6月。
\newcommand{\isacademicdegree}{false} % 是否为学术学位,true为学术学位,false为专业学位
\newcommand{\zhmentor}{亚当·斯密\ \ 教授} % 导师:亚当·斯密\ \ 教授 / 张三\ \ 副教授 / 李四\ \ 讲师 / 王五\ \ 助理教授
\newcommand{\enmentor}{Prof.\ Adam Smith} % 导师的英文名:Prof. Adam Smith / A.P. Zhang San / Lec. Li Si 
\newcommand{\zhkeywords}{信息不对称,偷懒,激励,博弈论} % 关键词:信息不对称,偷懒,激励,博弈论
\newcommand{\enkeywords}{Asymmetric Information, Laziness, Incentive, Game Theory} % 关键词的英文名:Asymmetric Information, Laziness, Incentive, Game Theory


% 自定义的配置
\usepackage{tikz}
\usetikzlibrary{arrows.meta, positioning, calc} % 加载 tikz 库
