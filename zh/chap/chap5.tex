\chapter{实证模型与结果}
\label{chap:empirical} % 确保已加载 tikz 包

本章旨在通过计量经济学方法,实证检验影响员工\zhquote{摸鱼}行为的关键因素,并评估第二、三章理论模型中提出的部分机制。我们将利用(虚构的)中国企业员工调查(China Enterprise Employee Survey, CEES)数据,构建计量模型进行分析。首先,阐述模型设定;其次,介绍数据来源与变量测量;再次,报告基准回归结果;最后,通过系数可视化和讨论,增强结果的可信度。

\section{模型设定}

考虑到员工的\zhquote{摸鱼}行为可能受到许多不随时间变化的个体固有特质(如个人性格、工作态度、固有能力等)以及企业层面固定特征的影响,这些因素若不加以控制,可能导致遗漏变量偏误。因此,本研究主要采用面板数据固定效应模型(Fixed Effects Model)来控制个体层面的非时变异质性。同时,模型也纳入了年份固定效应,以吸收宏观经济环境或普遍性技术冲击等随时间变化的共同因素。基准计量模型设定如下:

\begin{equation}
\label{eq:empirical_model}
\begin{aligned}
\text{ShirkingIndex}_{it} = & \beta_0 + \beta_1 \text{MonitoringIntensity}_{it} + \\
                            & \beta_2 \text{PerformancePayShare}_{it} + \\
                            & \beta_3 \text{TaskComplexity}_{it} + \\
                            & X'_{it} \omega + \alpha_i + \gamma_t + \varepsilon_{it}
\end{aligned}
\end{equation}

\noindent
其中,下标 \(i\) 代表个体员工,\(t\) 代表年份。\(\text{ShirkingIndex}_{it}\) 为因变量,衡量员工 \(i\) 在年份 \(t\) 的\zhquote{摸鱼}程度。\(\text{MonitoringIntensity}_{it}\) 代表企业对员工的监督强度。\(\text{PerformancePayShare}_{it}\) 表示绩效工资占员工总收入的比重。\(\text{TaskComplexity}_{it}\) 衡量员工承担工作任务的复杂性和自主性。\(X'_{it}\) 是一系列时变的个体和企业层面控制变量的向量,\(\omega\) 是对应的系数向量。\(\alpha_i\) 代表个体固定效应,控制所有不随时间改变的个体特征。\(\gamma_t\) 代表年份固定效应,控制所有年份共有的时间趋势。\(\varepsilon_{it}\) 是随机误差项。

本模型的核心在于估计系数 \(\beta_1\), \(\beta_2\), 和 \(\beta_3\)。根据理论预期,\(\beta_1\) 预计为负,因为更高的监督强度应能抑制\zhquote{摸鱼}行为。同样,\(\beta_2\) 预计为负,原因是更高比例的绩效工资与努力和产出挂钩,应能减少\zhquote{摸鱼}。\(\beta_3\) 的符号则具有不确定性。一方面,更复杂、自主性更高的工作可能激发员工的内在动机,从而减少\zhquote{摸鱼};另一方面,复杂工作往往更难监督,可能为\zhquote{摸鱼}提供更多空间。通过估计该模型,我们可以检验这些理论预测在(虚构的)数据中是否得到支持。

\section{数据与变量说明}

\subsection{数据来源}

本研究使用的数据来源于一项虚构的、模拟中国实际情况的大型追踪调查——\zhquote{中国企业员工调查(CEES)}。该调查(假设)于 2015 年启动基线调查,并于 2017、2019 和 2021 年进行了三次追访,覆盖了中国东、中、西部多个省份不同行业的企业及其员工。CEES 收集了详细的员工个人信息、工作特征、薪酬结构、工作态度以及企业层面的管理实践信息。本研究构建了一个非平衡面板数据集,包含在所有四次调查中至少被成功访问两次的员工。经过数据清理和筛选(例如,剔除关键变量缺失严重的样本),最终用于分析的样本包含 15,820 名员工,共计 48,550 个员工-年份观测值。

\subsection{变量测量}

首先,我们定义因变量{摸鱼指数 (\(\text{ShirkingIndex}_{it}\))},该变量旨在量化员工的\zhquote{摸鱼}程度。在 CEES 调查中(假设),我们利用一系列问题综合构建此指数。例如,询问员工在工作时间内从事非工作相关活动的频率(如浏览社交媒体、网购、处理私人事务等)、自我感觉的努力投入程度(相对于同事或自身最大可能)、是否存在故意拖延工作等行为。通过因子分析或简单加权平均,将这些信息合成为一个介于 0 到 10 之间的综合指数,数值越高表示\zhquote{摸鱼}程度越严重。

接下来是核心解释变量。第一项是{监督强度 (\(\text{MonitoringIntensity}_{it}\))},基于 CEES 中企业问卷部分关于管理实践的问题构建。例如,管理层对员工工作过程的直接观察频率、是否使用电子监控系统(如电脑活动监控、工位摄像头)、工作日志或报告的要求严格程度等。将这些信息标准化并合成为一个 0 到 1 之间的指数,数值越高表示监督强度越大。第二项是{绩效工资占比 (\(\text{PerformancePayShare}_{it}\))},根据员工问卷中关于其收入构成(固定工资、基于个人绩效的奖金、基于团队或公司绩效的奖金等)的回答计算得出。定义为绩效相关收入(奖金、佣金等)占其年度总税前收入的比例,取值范围为 [0, 1]。第三项为{任务复杂度 (\(\text{TaskComplexity}_{it}\))},基于员工对其工作任务的评价构建。涉及的问题可能包括:工作的重复性程度、解决问题时所需的创造性、工作自主权的大小、技能多样性要求等。同样合成为一个 1 到 5 的指数,数值越高表示任务越复杂、自主性越高。

最后是控制变量 (\(X'_{it}\))。为更准确地识别核心解释变量的影响,模型中加入了以下时变的控制变量:{对数工资 (\(\text{LogWage}_{it}\))},即员工年度总税前收入的自然对数,因为工资水平可能直接影响工作满意度和努力意愿;{工龄 (\(\text{Tenure}_{it}\))},即员工在当前企业的服务年限,可能与组织承诺、工作熟悉度等相关;{年龄 (\(\text{Age}_{it}\))} 和 {年龄平方 (\(\text{Age}^2_{it}\))},用以控制年龄对工作态度的非线性影响;{教育年限 (\(\text{Education}_{it}\))},即员工接受正规教育的总年数;{性别 (\(\text{Gender}_{i}\))},一个虚拟变量(1 代表男性,0 代表女性),尽管由于使用个体固定效应,性别等不随时间变化的变量本身无法被估计,但其与其他时变变量的交互影响仍可能存在;{企业规模 (\(\text{FirmSize}_{it}\))},即员工所在企业总人数的对数,因为企业规模可能影响组织结构、管理风格和监督难度;以及一系列{行业虚拟变量 (\(\text{IndustryDummies}_{i}\))}(同样受个体固定效应影响)。此外,个体固定效应 (\(\alpha_i\)) 控制了个体层面不随时间变化的因素(如能力、性格),年份固定效应 (\(\gamma_t\)) 控制了随时间变化的宏观因素。

\section{基准回归结果}\label{sec:empirical_baseline_results}

表 \ref{tab:fe_results} 展示了基于方程 (\ref{eq:empirical_model}) 的固定效应模型估计结果。模型(1)只包含核心解释变量,模型(2)加入了个人层面的控制变量(对数工资、工龄、年龄等),模型(3)进一步加入了企业层面的控制变量(企业规模)以及所有固定效应。所有模型均报告了聚类到个体层面的稳健标准误。

{\small % 应用 \small 字体
\begin{longtable}{p{5cm} p{2.2cm} p{2.2cm} p{2.2cm}} % 调整列宽
\caption{固定效应模型回归结果:影响摸鱼指数的因素} \label{tab:fe_results} \\
\toprule
& \multicolumn{3}{c}{摸鱼指数 (ShirkingIndex)} \\ % 添加因变量的跨列标题
\cmidrule(lr){2-4} % 在跨列标题下方添加部分横线
变量 & (1) & (2) & (3) \\
\midrule
\endfirsthead

\multicolumn{4}{c}%
{{续\tablename\ \thetable{}\quad 固定效应模型回归结果:影响摸鱼指数的因素}} \\ % 续表标题
\toprule
& \multicolumn{3}{c}{摸鱼指数 (ShirkingIndex)} \\ % 续页的因变量标题
\cmidrule(lr){2-4} % 续页的部分横线
变量 & (1) & (2) & (3) \\
\midrule
\endhead

\midrule
\multicolumn{4}{r@{}}{{续下页}} \\ % 续页提示
\endfoot

\bottomrule
% 使用与之前示例一致的脚注格式
\multicolumn{4}{@{}l}{\footnotesize 注:括号内为聚类到个体层面的稳健标准误。*, **, *** 分别表示在 10\%, 5\%, 1\% 的水平上显著。} \\ % 左对齐脚注
\multicolumn{4}{@{}l}{\footnotesize 所有模型均包含个体固定效应和年份固定效应。模型(3)为完整模型。} \\
\endlastfoot

% --- 表格主体内容开始 ---
{核心解释变量:} & & & \\
\quad 监督强度 (MonitoringIntensity) & -1.523*** & -1.285*** & -1.150*** \\
& (0.188) & (0.201) & (0.205) \\
\quad 绩效工资占比 (PerformancePayShare) & -2.105*** & -1.850*** & -1.782*** \\
& (0.255) & (0.260) & (0.263) \\
\quad 任务复杂度 (TaskComplexity) & 0.312** & 0.258* & 0.280** \\
& (0.124) & (0.130) & (0.133) \\
\\ % 添加空行分隔核心变量和控制变量
{控制变量:} & & & \\
\quad 对数工资 (LogWage) & & 0.085 & 0.070 \\
& & (0.060) & (0.061) \\
\quad 工龄 (Tenure) & & -0.021** & -0.018* \\
& & (0.009) & (0.010) \\
\quad 工龄平方 (TenureSq) & & 0.0003 & 0.0002 \\
& & (0.0002) & (0.0002) \\
\quad 年龄 (Age) & & -0.035 & -0.030 \\
& & (0.025) & (0.026) \\
\quad 年龄平方 (AgeSq / 100) & & 0.040 & 0.038 \\
& & (0.030) & (0.031) \\
\quad 教育年限 (Education) & & -0.050 & -0.045 \\
& & (0.038) & (0.039) \\
\quad 企业规模 (LogFirmSize) & & & -0.120* \\
& & & (0.070) \\
\quad 常数项 (\_cons) & 4.850*** & 5.210*** & 5.530*** \\
& (0.310) & (0.450) & (0.480) \\
\midrule % 主体系数与下方说明之间的横线
个体固定效应 & 是 & 是 & 是 \\
年份固定效应 & 是 & 是 & 是 \\
\midrule % 上方说明与统计量之间的横线
观测值 (N) & 48550 & 48550 & 48550 \\
组内 R$^2$ (R2 within) & 0.085 & 0.102 & 0.105 \\
组间 R$^2$ (R2 between) & 0.150 & 0.180 & 0.183 \\
总体 R$^2$ (R2 overall) & 0.120 & 0.145 & 0.148 \\
F 统计量 (F statistic) & 85.3*** & 65.2*** & 62.5*** \\
员工数 & 15820 & 15820 & 15820 \\
% --- 表格主体内容结束 ---

\end{longtable}
} % 结束 \small 字体范围

从完整模型(列 3)的结果来看,{监督强度 (\(\text{MonitoringIntensity}\))} 的系数显著为负 (-1.150, p<0.001),这与理论预期一致。结果表明,在控制其他因素后,企业监督强度的提高能够显著降低员工的\zhquote{摸鱼}指数。具体而言,监督强度指数每增加一个单位(从0到1),员工的摸鱼指数平均下降约 1.15 个点。{绩效工资占比 (\(\text{PerformancePayShare}\))} 的系数也显著为负 (-1.782, p<0.001),同样符合理论预期。绩效薪酬在总收入中占比的提高,将员工的利益与工作成果更紧密地联系起来,从而有效减少了\zhquote{摸鱼}行为。占比每提高 10 个百分点,摸鱼指数平均下降约 0.18 个点。与前两者不同,{任务复杂度 (\(\text{TaskComplexity}\))} 的系数显著为正 (0.280, p<0.05)。这提示我们,在控制其他因素后,任务复杂度(和自主性)的提高反而与更高的\zhquote{摸鱼}指数相关,可能反映了复杂任务更难被精确监督,为员工提供了更多\zhquote{在岗磨洋工}的机会,其负面效应超过了潜在的内在激励效应。

控制变量方面,工龄似乎与较低的\zhquote{摸鱼}程度相关,而企业规模的增大也倾向于降低\zhquote{摸鱼}指数(可能因为大企业有更规范的管理)。对数工资、年龄、教育等变量在本模型中不显著,这可能是因为个体固定效应已经吸收了这些变量的大部分解释力,或者是它们对\zhquote{摸鱼}行为的影响确实不显著。

总体而言,基准回归结果为理论模型中关于监督、激励(绩效工资)对约束\zhquote{摸鱼}行为有效性的假设提供了(虚构的)实证支持。同时,任务复杂度对\zhquote{摸鱼}的正向影响也揭示了管理实践中需要注意的权衡。

\section{结果稳健性与讨论}\label{sec:empirical_robustness}

为了更直观地展示核心解释变量的影响,并探讨结果的稳健性,我们绘制了基准模型(表 \ref{tab:fe_results} 列 3)中核心解释变量的系数及其 95\% 置信区间(见图 \ref{fig:coeff_plot})。

\begin{figure}[htp]
    \centering
    \begin{tikzpicture}[scale=2.5, every node/.style={scale=1.6}]
        % 定义坐标轴
        \coordinate (origin) at (0,0);
        \draw [->, >=Latex] (-3,0) -- (3,0) node [below left, font=\tiny] {系数估计值}; % X axis
        \draw [->, >=Latex] (0,-0.5) -- (0,3.5) node [below left, font=\tiny] {}; % Y axis (for labels)
        \draw [dashed, thin, gray] (0,0) -- (0,3); % Zero line

        % 定义变量位置和标签
        \node[left=1cm of origin, yshift=2.5cm, anchor=east, font=\scriptsize] (label1) {监督强度};
        \node[left=1cm of origin, yshift=1.5cm, anchor=east, font=\scriptsize] (label2) {绩效工资占比};
        \node[left=1cm of origin, yshift=0.5cm, anchor=east, font=\scriptsize] (label3) {任务复杂度};

        % 绘制点估计和置信区间
        % 监督强度: est = -1.150, se = 0.205, ci = [-1.552, -0.748] (approx 1.96*se)
        \coordinate (est1) at (-1.150, 2.5cm);
        \draw [thick, blue] (est1) circle (1.5pt);
        \draw [blue] ($(est1)-(1.96*0.205, 0)$) -- ($(est1)+(1.96*0.205, 0)$);
        \draw [blue] ($(est1)-(1.96*0.205, 0)$) + (0, -0.1) -- +(0, 0.2);
        \draw [blue] ($(est1)+(1.96*0.205, 0)$) + (0, -0.1) -- +(0, 0.2);

        % 绩效工资占比: est = -1.782, se = 0.263, ci = [-2.297, -1.267] (approx 1.96*se)
        \coordinate (est2) at (-1.782, 1.5cm);
        \draw [thick, blue] (est2) circle (1.5pt);
        \draw [blue] ($(est2)-(1.96*0.263, 0)$) -- ($(est2)+(1.96*0.263, 0)$);
        \draw [blue] ($(est2)-(1.96*0.263, 0)$) + (0, -0.1) -- +(0, 0.2);
        \draw [blue] ($(est2)+(1.96*0.263, 0)$) + (0, -0.1) -- +(0, 0.2);

        % 任务复杂度: est = 0.280, se = 0.133, ci = [0.019, 0.541] (approx 1.96*se)
        \coordinate (est3) at (0.280, 0.5cm);
        \draw [thick, red] (est3) circle (1.5pt); % Use red for positive effect
        \draw [red] ($(est3)-(1.96*0.133, 0)$) -- ($(est3)+(1.96*0.133, 0)$);
        \draw [red] ($(est3)-(1.96*0.133, 0)$) + (0, -0.1) -- +(0, 0.2);
        \draw [red] ($(est3)+(1.96*0.133, 0)$) + (0, -0.1) -- +(0, 0.2);

        % 添加 X 轴刻度
        \foreach \x in {-2,-1.5,-1,-0.5,0,0.5,1,1.5,2} {
            \draw (\x, -0.1) -- (\x, 0.1) node[below, font=\tiny] {\x};
        }

    \end{tikzpicture}
    \caption{核心解释变量的系数估计值与 95\% 置信区间}
    \label{fig:coeff_plot}
\end{figure}

图 \ref{fig:coeff_plot} 清晰地显示,监督强度和绩效工资占比的置信区间完全位于零的左侧,表明它们的负向影响是统计上稳健的。任务复杂度的置信区间则位于零的右侧,但接近于零,显示其正向影响虽然显著(在 5\% 水平上),但效应大小相对前两者较小,且稳健性可能稍弱。

此外,我们还进行了(或声称进行了)一些额外的稳健性检验(结果未在表中完全展示)。首先,我们{替换了因变量},使用其他衡量\zhquote{摸鱼}的代理变量,例如自我报告的\zhquote{工作努力程度}(反向编码)或(假设可得的)工作时间内访问非工作相关网站的时长(Cyberloafing Hours)作为因变量,核心解释变量的系数符号和显著性基本保持不变。其次,进行了{子样本分析},将样本按照行业(如制造业 vs 服务业)或企业所有制(国有 vs 非国有)进行分组回归。结果发现监督强度和绩效工资占比的负向效应在不同子样本中普遍存在,但效应大小可能有所差异。例如,在任务更标准化的制造业中,监督的效应可能更强;在市场化程度更高的非国有企业中,绩效工资的激励效果可能更显著。任务复杂度的正向效应在服务业或知识密集型工作中可能更明显。最后,我们{考虑了内生性问题}。虽然固定效应模型缓解了部分遗漏变量问题,但仍可能存在时变遗漏变量或反向因果关系(例如,\zhquote{摸鱼}严重的员工可能更容易被置于低绩效工资比例的岗位上)。我们尝试使用(虚构的)工具变量(如公司层面引入新监控技术的决策,作为监督强度的工具变量)进行初步探索,结果(假设)支持了基准回归的结论,但工具变量的有效性需要更严格的论证。

{讨论与启示}:
本章的实证分析(尽管基于虚构数据)为理解员工\zhquote{摸鱼}行为的影响因素提供了一些经验证据。结果强调了外部激励和监督机制在约束机会主义行为中的重要作用,这与委托代理理论的核心观点一致。企业可以通过加强监督(提高被发现的概率或惩罚预期)和优化薪酬结构(提高绩效关联度)来有效降低员工的\zhquote{摸鱼}水平。

然而,任务复杂度的正向效应提示管理者需要关注工作的内在属性与管理方式的匹配。对于复杂性、创造性要求高的工作,过度依赖传统监督可能效果有限甚至适得其反(可能破坏内在动机,尽管本研究未直接测量)。此时,可能需要探索更侧重结果导向、信任和赋权的激励与管理方式。

当然,本研究也存在局限性。首先,\zhquote{摸鱼}行为的测量依赖于(虚构的)问卷数据,可能存在社会期望偏差或主观性。其次,内生性问题难以完全排除。未来的研究可以尝试利用更客观的数据(如工作日志、数字足迹)或更可靠的准自然实验设计来进一步验证这些发现。