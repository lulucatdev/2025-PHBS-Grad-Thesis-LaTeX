\begin{cabstract}
	\addcontentsline{toc}{chapter}{摘要}

	在现代工作环境中,“摸鱼”现象,即员工在工作时间从事非工作相关事务或降低工作投入度的行为,日益普遍。这种现象不仅影响组织效率,也反映了雇佣关系中深刻的经济学问题。本文从{信息经济学}的视角出发,基于{委托代理理论}和{博弈论},探讨在{信息不对称}环境下,员工如何选择最优的“{摸鱼}”策略,以及雇主如何设计最优的{激励}与{监督}机制。研究旨在揭示该行为的内在经济逻辑,并为理解和管理现代劳动关系提供理论依据与(模拟的)实证参考。

	本文首先构建了一个简约的理论模型。在该模型中,风险中性的雇主无法直接观测风险中性雇员的努力程度,但可以通过设定最低努力标准、支付固定工资,并辅以概率性{监督}和惩罚机制来激励雇员。模型推导表明,雇员的最优决策呈现门槛效应:只有当预期惩罚(监督概率乘以惩罚力度)超过达到最低努力标准的成本时,雇员才会选择恰好达到该标准;否则,将选择完全不努力(最大化“摸鱼”)。进而,模型分析了雇主在权衡产出收益、工资支付和监督成本后,如何选择最优的最低努力要求和监督概率。

	为检验理论模型的启示,本文利用(虚构的)“中国企业员工调查(CEES)”面板数据,构建了固定效应计量模型进行实证分析。结果显示:企业{监督}强度的提高和{绩效工资}占比的增加,均与员工“摸鱼”指数呈显著负相关,验证了{监督}和{激励}对约束机会主义行为的有效性;而任务复杂度的提高则与“摸鱼”指数呈正相关,可能反映了复杂工作更难被有效监督的特性。

	研究结合了理论推导与(虚构的)实证检验,深化了对{信息不对称}下“摸鱼”行为经济根源的理解。研究结果为企业管理者设计更有效的{激励}机制(如平衡绩效薪酬与监督投入)和管理策略(如针对不同复杂度工作采取差异化管理方式)提供了启示。本文强调,理解“摸鱼”的经济学原理,并非鼓励该行为,而是旨在通过科学分析促进更和谐、高效的劳资关系。
\end{cabstract}

% vim:ts=4:sw=4
