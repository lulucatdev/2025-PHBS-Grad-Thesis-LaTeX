\chapter{致谢}

回首这段探索\zhquote{摸鱼}经济学奥秘的旅程,感慨万千。若非众位师友襄助,恐怕我的\zhquote{最优偷懒策略}早已偏离了完成论文的正轨,陷入了某种难以收敛的\zhquote{次优均衡}。

首先,我要向我的导师,亚当·斯密教授,致以最诚挚的谢意。斯密老师以其深厚的经济学洞察力,如同那只\zhquote{看不见的手},巧妙地引导着我的研究方向,在我在信息的迷雾中探索\zhquote{激励}与\zhquote{监督}的边界时,给予了关键性的指导和无尽的耐心。尤其在我试图将委托代理理论应用于分析自身拖延行为时,老师总能一语中的,促使我回归学术正途。

感谢学院的其他老师们,他们的课程和讨论为我打下了坚实的理论基础,尤其是在{博弈论}和{信息经济学}方面,使我能够构建和分析第三、四章中的模型。

感谢与我一同奋斗的同学们和朋友们。无数次的讨论(有时甚至是在咖啡馆里进行的\zhquote{非正式监督}),从第一章的引论构思,到第二章文献的梳理,再到第五章(虚构却煞费苦心的)实证结果的解读,你们的智慧火花和质疑精神,是推动我前进的重要动力。特别感谢那15,820位(存在于硬盘和想象中的)CEES受访者,感谢你们对\zhquote{摸鱼指数}的坦诚\zhquote{汇报}。

当然,要特别感谢我的家人。在这场旷日持久的\zhquote{重复博弈}中,你们无条件的爱、支持和包容,是我能坚持下来的最终\zhquote{参与约束}。你们的鼓励是我在面对研究瓶颈和自我怀疑时,重新调整\zhquote{努力成本函数}的强大动力。

最后,我要感谢这段研究经历本身。深入探究\zhquote{摸鱼}背后的经济学原理, paradoxically,是一个极其耗费心力、杜绝\zhquote{摸鱼}的过程。希望这项研究不仅能为理解现代劳动关系提供些许视角(如第六章所展望),也算是我个人在这段学术征途中,交出的一份没有\zhquote{过度偷懒}的答卷。

于此,谨向所有给予我帮助和启发的人们,再次表示最衷心的感谢!