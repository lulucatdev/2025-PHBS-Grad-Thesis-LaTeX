\chapter{理论模型构建}
\label{chap:model}

本章旨在构建一个理论模型,刻画信息不对称下雇员的\zhquote{摸鱼}(即选择努力程度)行为以及雇主的最优应对策略。模型将借鉴委托代理理论和博弈论的基本框架。

\section{模型基本设定}
\label{sec:model_setup}

考虑一个单一周期(single-period)的雇佣关系,参与者包括一个风险中性(risk-neutral)的雇主(委托人,Principal, P)和一个同样是风险中性的雇员(代理人,Agent, A)。雇员拥有保留效用(reservation utility) $\bar{U}$,代表其接受雇佣关系的最低效用水平,为简化分析,我们将其标准化为 $\bar{U} = 0$。

博弈顺序如下:
\begin{enumerate}
    \item  雇主设计并提供一份雇佣契约给雇员。
    \item  雇员决定是否接受契约。如果拒绝,雇员获得保留效用 $\bar{U}=0$,雇主获得0利润。如果接受,博弈继续。
    \item  雇员选择其努力程度 $e$。努力程度 $e \ge 0$。同时,雇员付出努力会产生相应的成本。
    \item  产出 $q$ 实现,支付根据契约执行。
\end{enumerate}

核心假设是信息不对称:雇主无法直接观测到雇员选择的努力程度 $e$。但是,雇主可以通过监督机制(monitoring)或观察最终产出 $q$ 来间接推断或影响雇员的努力选择。

\subsection{雇员的努力与成本}

雇员选择努力程度 $e$。我们假设努力程度是一个连续变量,$e \in [0, \infty)$。付出努力会给雇员带来负效用(成本)。我们用成本函数 $c(e)$ 来表示,并假设该函数具有以下性质:
\begin{itemize}
    \item $c(0) = 0$:不付出努力则没有成本。
    \item $c'(e) > 0$ for $e > 0$:努力成本随努力程度的增加而增加(边际成本为正)。
    \item $c''(e) > 0$:努力的边际成本递增(成本函数是严格凸函数)。
\end{itemize}
A commonly used form for the cost function is $c(e) = \frac{k}{2}e^2$, where $k > 0$ 是一个成本参数,反映了努力的困难程度。

雇员的\zhquote{摸鱼}程度可以视为其选择的努力水平 $e$ 相对于某个基准(例如,雇主期望的水平或最大可能水平)的偏离。在本模型中,我们直接分析雇员的最优努力 $e$ 的选择。

\subsection{产出与支付}

为简化模型,我们首先考虑一个确定性产出(deterministic output)的情形,即产出 $q$ 完全由雇员的努力程度决定:
\begin{equation}
q = e
\end{equation}
这意味着雇主可以通过观察产出 $q$ 来完美推断努力 $e$。在这种情况下,不存在信息不对称,雇主可以通过设计一个强制特定努力水平 $e^*$ 并支付相应工资的契约来达到最优。例如,规定如果 $q=e^*$ 则支付 $w$,否则支付0或进行惩罚。只要 $w - c(e^*) \ge 0$,雇员就会接受并选择 $e=e^*$。

然而,现实中产出往往受到随机因素的影响。一个更现实的设定是随机产出(stochastic output),例如:
\begin{equation}
q = e + \epsilon
\end{equation}
其中 $\epsilon$ 是一个均值为0的随机噪声项(例如,服从正态分布 $N(0, \sigma^2)$)。在这种情况下,即使观察到产出 $q$,雇主也无法完全确定雇员的努力 $e$,信息不对称问题凸显。

契约可以基于可观测的变量,如产出 $q$ 或通过监督获得的信息。常见的契约形式包括:
\begin{itemize}
    \item 固定工资契约(Fixed Wage Contract):无论产出如何,支付固定工资 $w$。
    \item 线性绩效工资契约(Linear Performance Pay Contract):工资 $w = s + bq$,其中 $s$ 是固定部分,$b$ 是基于产出的佣金率(bonus rate)。
    \item 基于监督的契约(Monitoring-Based Contract):雇主以一定概率 $p$ 进行监督。如果监督发现雇员行为不符合要求(例如,努力低于某个标准 $e^*$),则进行惩罚。
\end{itemize}

\section{雇员行为分析:固定工资与监督}
\label{sec:agent_behavior_monitor}

现在我们分析在一个相对简单的契约结构下,雇员如何选择其最优努力程度。考虑雇主提供一份包含固定工资 $w$ 和监督惩罚机制的契约。

具体地,雇主以概率 $p \in [0, 1]$ 对雇员进行监督。监督是完美的,一旦实施,就能准确观测到雇员的实际努力程度 $e$。契约规定了一个最低努力标准 $e_{min} \ge 0$。如果雇员被监督到,且其努力程度 $e < e_{min}$,则雇员需要承担一个惩罚 $F > 0$(例如,罚款、扣减工资等)。如果 $e \ge e_{min}$,或者雇员没有被监督到(概率为 $1-p$),则不会受到惩罚。

假设雇员是风险中性的,其目标是最大化其期望效用。雇员的效用来自于获得的工资,减去努力成本,再减去可能的期望惩罚。给定契约 $(w, p, e_{min}, F)$,雇员选择努力 $e$ 来最大化:
\begin{equation}
E[U(e)] = w - c(e) - p \cdot \mathbb{I}(e < e_{min}) \cdot F
\end{equation}
其中 $\mathbb{I}(\cdot)$ 是指示函数,当条件成立时取值为1,否则为0。

雇员的最优决策如下:
\begin{enumerate}
    \item  如果选择 $e \ge e_{min}$**:此时 $\mathbb{I}(e < e_{min}) = 0$,雇员不会受到惩罚。其效用为 $U_1(e) = w - c(e)$。为了最大化效用,雇员会选择满足 $e \ge e_{min}$ 约束下的最低成本努力,即选择 $e = e_{min}$。此时的效用为 $U_1^* = w - c(e_{min})$。

    \item  如果选择 $e < e_{min}$**:此时 $\mathbb{I}(e < e_{min}) = 1$,雇员面临被发现并惩罚的风险。其期望效用为 $E[U_2(e)] = w - c(e) - pF$。为了最大化这个期望效用,雇员会选择成本最低的努力,即 $e=0$(假设 $e=0$ 是允许的最低努力)。此时的期望效用为 $E[U_2^*] = w - c(0) - pF = w - pF$。
\end{enumerate}

雇员将在上述两种情况中选择能带来更高(期望)效用的那个:
雇员会选择 $e = e_{min}$ 当且仅当
\begin{equation}
U_1^* \ge E[U_2^*]
\end{equation}
\begin{equation}
w - c(e_{min}) \ge w - pF
\end{equation}
\begin{equation}
pF \ge c(e_{min})
\end{equation}

{结论}:在该契约下,雇员的最优努力选择 $e^*$ 是:
\begin{equation}
e^* = \begin{cases} e_{min} & \text{if } pF \ge c(e_{min}) \\ 0 & \text{if } pF < c(e_{min}) \end{cases}
\end{equation}

这个结果直观地说明了:只有当预期的惩罚(被发现的概率 $p$ 乘以惩罚力度 $F$)足够大,能够超过达到最低努力标准 $e_{min}$ 所需的成本 $c(e_{min})$ 时,雇员才会被激励去达到这个标准。否则,雇员宁愿选择完全不努力($e=0$),并承担 $pF$ 的预期惩罚成本,因为这样做可以节省 $c(e_{min})$ 的努力成本。

这个简单的模型揭示了监督和惩罚在约束\zhquote{摸鱼}行为中的作用,但也显示了其局限性:它只能激励雇员达到最低标准 $e_{min}$,而无法激励更高的努力水平。并且,努力的选择呈现一个\zhquote{全有或全无}(相对于 $e_{min}$ 而言)的特征。

% 下一节将分析雇主如何设计最优契约